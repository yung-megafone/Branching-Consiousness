\documentclass{article}
\usepackage{amsmath, amssymb, graphicx, hyperref, cite}
\usepackage[a4paper, margin=1in]{geometry}
\hypersetup{colorlinks=true, linkcolor=blue, citecolor=blue, urlcolor=blue}

\title{Wave-Form Consciousness and Reality Selection: A Quantum Expansion of the Branching Repository Reality Theory}
\author{Branden Anasenes \href{mailto:b.anasenes@pm.me}{b.anasenes@pm.me} \href{https://orcid.org/0009-0005-1580-9026}{ORCID: 0009-0005-1580-9026}}
\date{March 9, 2025}

\begin{document}

\maketitle

\begin{abstract}
This paper presents a formal mathematical framework for the \textbf{Branching Repository Reality Theory (BRRT)}, which conceptualizes consciousness as a quantum-inspired wave function interacting with reality through probabilistic superposition and selective collapse. We refine the probability model for reality selection, integrating established findings from quantum mechanics, neuroscience, and cognitive psychology \cite{schrodinger1935present, fisher2015quantum, carhart2012neural}. Experimental methodologies—including neuroimaging, psychedelic-induced time dilation studies, and computational modeling—are explored to provide empirical support for the theory. Importantly, BRRT offers \textbf{falsifiable predictions} regarding consciousness, perception anomalies, and the quantum-like properties of cognition. While BRRT remains a theoretical framework, we discuss potential limitations such as quantum decoherence constraints, the applicability of the Many-Worlds Interpretation (MWI) to cognition, and the distinction between probability-based consciousness states and multiversal branching. 

The implications of BRRT challenge conventional notions of consciousness, free will, and reality perception, offering a paradigm shift in our understanding of cognition and the multiversal structure of experience. \textbf{If validated, BRRT may redefine cognition as an active participant in shaping the quantum probability landscape.}
\end{abstract}

\section{Introduction}
The relationship between consciousness and reality remains a profound scientific question. Quantum mechanics suggests that particles exist in a superposition of states until measured, collapsing into a definite outcome \cite{everett1957relative}. \textbf{The Branching Repository Reality Theory (BRRT)} extends this principle to human cognition, proposing that consciousness functions as a probability-weighted wave function that determines which reality is experienced. However, unlike the Many-Worlds Interpretation (MWI), which describes physical branching of quantum states, BRRT does not posit literal multiversal splitting. Instead, BRRT suggests a probability-weighted perception of reality, where selection occurs based on neural and environmental interactions.

Unlike classical probability models, which assume deterministic or stochastic neural activity with pre-defined transition probabilities, BRRT proposes a dynamic selection process where reality-state probabilities evolve based on neural wave function interactions. If experimentally validated, this would suggest that cognition operates beyond classical determinism, possibly influencing subjective reality at a fundamental level. If quantum effects are confirmed in cognition, this would challenge existing classical models and offer a novel perspective on free will and reality perception.

This theory integrates quantum probability models with cognitive neuroscience, postulating that:
\begin{itemize}
    \item Consciousness behaves as a probabilistic wave function dynamically interacting with external reality.
    \item Reality "branches" not through discrete timeline splits but via quantum probability interference, allowing fluid transitions between states.
    \item Phenomena such as \textbf{d\'ej\`a vu, the Mandela Effect, and altered states of consciousness} may be explained through wave interference and probability selection mechanisms \cite{psychedelic2020research}.
    \item Time perception and near-death experiences may correspond to transitions between probability-weighted reality states rather than deterministic sequences \cite{carhart2012neural}.
\end{itemize}

We develop a formal mathematical foundation for BRRT and propose experimental methodologies to test its validity while acknowledging key theoretical limitations.

\section{Mathematical Model of Consciousness and Reality Selection}

Following the formalism of Schr\"odinger's equation \cite{schrodinger1935present}, we define consciousness as a wave function $\Psi_c$, evolving as:

\begin{equation}
    \hat{H} \Psi_c = i\hbar \frac{\partial \Psi_c}{\partial t}
\end{equation}

where:
\begin{itemize}
    \item $\Psi_c$ represents the consciousness wave function.
    \item $\hat{H}$ is the Hamiltonian operator governing the system’s evolution.
    \item $i\hbar$ is the imaginary unit times Planck’s reduced constant.
    \item $\frac{\partial \Psi_c}{\partial t}$ describes the rate of change of consciousness over time.
\end{itemize}

To model the probability of consciousness selecting a particular reality $R_n$, we define:

\begin{equation}
    \mathbb{P}(R_n) = |\langle R_n | \Psi_c(t) \rangle|^2
\end{equation}

where:
\begin{itemize}
    \item $\mathbb{P}(R_n)$ is the probability of experiencing reality $R_n$ at time $t$.
    \item $\langle R_n | \Psi_c(t) \rangle$ is the inner product of the consciousness wave function and the reality state.
\end{itemize}

This suggests that reality selection follows a probability amplitude function, where transitions between possible realities are influenced by neural coherence, cognitive expectation, and external perturbations. The total probability across all potential reality states must satisfy the normalization condition:

\begin{equation}
    \sum_{n} \mathbb{P}(R_n) = 1
\end{equation}

However, challenges such as \textbf{quantum decoherence in biological systems} \cite{tegmark2015consciousness} must be considered, as rapid decoherence could limit the extent to which quantum processes influence cognition. Counterarguments suggest that certain biological systems, such as nuclear spin interactions in neurons \cite{fisher2015quantum}, may enable sustained coherence despite classical decoherence constraints.

\section{Experimental Validation and Future Work}

Future studies should explore:
\begin{itemize}
    \item Neuroimaging studies (EEG, fMRI) to analyze probability-based perception shifts and test for anomalies that align with BRRT predictions.
    \item Quantum coherence experiments to determine if neural microtubules or nuclear spin interactions retain quantum properties \cite{hameroff2014consciousness}.
    \item AI-based models approximating probabilistic decision-making to compare with human cognition and test whether probability-weighted selection can be computationally modeled with quantum-inspired neural networks. However, due to entropy constraints, AI may be unable to fully simulate BRRT’s cognitive model.
    \item Computational models to evaluate whether BRRT differs from classical probability models in cognitive science \cite{tegmark2015consciousness}.
    \item If EEG patterns demonstrate probability-weighted transitions distinct from classical noise, this would strengthen BRRT’s claims. Conversely, an inability to detect such patterns may indicate a need for alternative models.
    \item Psychedelic-induced time dilation may serve as a macroscopic analogy for probability-weighted consciousness selection rather than direct evidence of quantum processes. However, if EEG/fMRI studies reveal statistical correlations between psychedelic states and non-classical probability distributions, this could provide indirect support for BRRT.
\end{itemize}

\section{Conclusion}
The BRRT framework provides a novel perspective on consciousness as a quantum-like process that dynamically selects reality states. By integrating quantum mechanics, neuroscience, and cognitive psychology, BRRT offers testable predictions regarding perception anomalies, altered states, and subjective time distortions. However, its feasibility is constrained by potential quantum decoherence effects and the lack of direct empirical validation. Future research must refine experimental validation, address theoretical limitations, and explore computational models simulating consciousness-wave interactions.

\bibliographystyle{unsrt}
\bibliography{brrt_references}

\end{document}