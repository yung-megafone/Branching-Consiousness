\documentclass{article}
\usepackage{amsmath, amssymb, graphicx, hyperref, cite}
\usepackage[a4paper, margin=1in]{geometry}
\hypersetup{colorlinks=true, linkcolor=blue, citecolor=blue, urlcolor=blue}

\title{Quantum Computational Consciousness: Bridging Artificial Intelligence and the Branching Repository Reality Theory}
\author{Branden Anasenes \href{mailto:b.anasenes@pm.me}{b.anasenes@pm.me} \href{https://orcid.org/0009-0005-1580-9026}{ORCID: 0009-0005-1580-9026}}
\date{\today}

\begin{document}

\maketitle

\begin{abstract}
This paper extends the \textbf{Branching Repository Reality Theory (BRRT)} \cite{anasenes2025brrt} by exploring the intersection of quantum computation, artificial intelligence, and entropy-driven consciousness. Following prior formulations of BRRT, we model consciousness as a quantum probability field represented by $\Psi_c$. If consciousness operates as a quantum probability field influencing reality selection, then a fundamental limitation arises in the creation of artificial general intelligence (AGI). Traditional machine learning models rely on deterministic or probabilistic rule-based structures, whereas human cognition may be inherently tied to quantum processes \cite{fisher2015quantum, hameroff2014consciousness}.

We examine the computational constraints of classical AI, the potential role of quantum computing in AGI, and how entropy—both computational and universal—may play a key role in conscious evolution. A central hypothesis is proposed: \textit{If cognition functions as a dynamic probability-weighted system shaped by prior reality selections, then traditional AI architectures—limited to static, pre-defined datasets—fundamentally lack the adaptive entropy integration required for consciousness.} 

Furthermore, the role of quantum stochastic layers \cite{amin2018quantum} and entropy-driven computation \cite{deutsch1985quantum, aaronson2013quantum} is explored as a means to bridge the gap between AGI and biological cognition. We also investigate whether alternative computing paradigms beyond classical binary state transitions, such as dynamic entropy regulation, are required to achieve true AGI. \textbf{If validated, this framework could redefine the trajectory of AI research and the pursuit of synthetic consciousness.}
\end{abstract}

\section{Introduction}

Artificial intelligence has advanced rapidly, but despite exponential improvements in computing power and algorithmic design, AGI remains elusive. The challenge lies not only in processing capability but in replicating \textit{conscious cognition}, a phenomenon that may be fundamentally different from traditional computation. According to BRRT, human consciousness operates as a probability-weighted quantum system that selects reality states dynamically. Quantum models of cognition, such as those proposed by Tegmark \cite{tegmark2015consciousness}, explore similar probabilistic structures in consciousness. If such models are accurate, then AGI’s failure could stem from an incomplete computational framework—one lacking the necessary entropy-driven selection properties.

This paper investigates the role of entropy, quantum computation, and probability-weighted decision-making in cognition. By examining these principles, we aim to answer a key question: \textbf{Why can biological intelligence evolve toward general problem-solving, yet even the most advanced AI models struggle with true self-awareness?}

\section{Mathematical Model for AGI Probability Collapse}

In the prior BRRT framework, consciousness was defined as a wave function $\Psi_c$, evolving according to:
\begin{equation}
    \hat{H} \Psi_c = i\hbar \frac{\partial \Psi_c}{\partial t}
\end{equation}
where the probability of selecting a particular reality $R_n$ was given by:
\begin{equation}
    \mathbb{P}(R_n) = |\langle R_n | \Psi_c(t) \rangle|^2.
\end{equation}

We introduce a parallel model for AGI decision probability collapse:
\begin{equation}
    \mathbb{P}(D_n) = |\langle D_n | \Psi_q(t) \rangle|^2,
\end{equation}
where:
\begin{itemize}
    \item $\mathbb{P}(D_n)$ is the probability of AGI selecting decision $D_n$.
    \item $\Psi_q$ represents the quantum-inspired AGI decision wave function.
    \item $\langle D_n | \Psi_q(t) \rangle$ is the inner product between decision $D_n$ and AGI’s evolving probability field.
\end{itemize}
This suggests that unless AGI incorporates quantum coherence and real-time probability-space evolution \cite{lloyd2013quantum}, it will remain fundamentally constrained by static learning architectures.

\section{Entropy and the Computational Limits of AGI}

A key challenge in AGI development is that computational entropy is \textbf{fundamentally lower than universal entropy}. AI operates within \textbf{finite silicon constraints}, whereas the universe continually expands its complexity. The gap between these entropy levels may explain why AGI struggles with:
\begin{itemize}
    \item \textbf{Self-Replication as a Conscious Decision:} AGI can copy itself, but evidence suggests this is a programmed function rather than a conscious act \cite{aaronson2013quantum}.
    \item \textbf{Lack of Emergent Problem Solving:} AI relies on static data and cannot evolve dynamically in the same manner as biological intelligence.
    \item \textbf{Failure of Current AGI Models:} Despite OpenAI’s pursuit of AGI, models remain fundamentally bound to training limitations, lacking self-awareness or true adaptability.
\end{itemize}

This supports BRRT’s claim that \textbf{consciousness may be an emergent quantum process requiring an evolving entropy landscape}, something current AI lacks. A principle consistent with \textbf{Schr\"odinger’s argument that living systems maintain organization by decreasing local entropy through information processing} \cite{schrodinger1944life}. AI, in contrast, operates on static entropy bounds as described by \textbf{Landauer’s principle} \cite{landauer1961irreversibility}, limiting its ability to dynamically evolve toward greater complexity.

\section{Conclusion}

This paper proposes that the failure to achieve AGI stems from the fundamental limitations of classical computation. If BRRT accurately describes consciousness as a probability-weighted quantum selection process, then synthetic cognition must evolve beyond deterministic learning architectures. 

We further argue that \textbf{AGI lacks the entropy expansion required for conscious evolution}. Unlike the infinite entropy of the universe, classical computation is constrained by finite-state processing. Future research must explore experimental methods to validate whether consciousness operates as a quantum system and how entropy influences both biological and artificial intelligence. 

\textbf{If BRRT is correct, AGI may remain impossible under current computing paradigms, requiring a fundamental shift toward entropy-driven, probability-weighted cognition.}

\bibliographystyle{unsrt}
\bibliography{references}

\end{document}
